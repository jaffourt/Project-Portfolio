%-------------------------------------------------------------------------------
%	SECTION TITLE
%-------------------------------------------------------------------------------
\cvsection{Professional Experience}

%-------------------------------------------------------------------------------
%	CONTENT
%-------------------------------------------------------------------------------

\begin{cventries}

%-------------------------------------------------------------------------------
%	JOB
%-------------------------------------------------------------------------------

  \blankcventry
  {Research Associate} % Degree
  {Massachusetts Institute of Technology} % Institution
  {May 2019 - Present} % Location
  {} % Date(s)

  \projectentry
  {Task Representations in Neural Networks}
  {\href{https://github.com/jaffourt/Project-Portfolio/tree/main/Project2}{\faicon{github}}} % Location
  {
    \begin{cvitems} % Description(s) of tasks/responsibilities
      \item {Led a research project examining task representations of \textbf{natural language} and memory in
      recurrent neural networks (RNN) using multi-task learning; this project addressed previously unexplained
      functional organization of RNNs in \textbf{machine learning} and \textbf{AI} research}
      \item {Built generative natural language processing models and trained for representation learning in a
      \textbf{PyTorch} codebase}
      \item {Implemented statistical analyses on recurrent units in \textbf{R} and \textbf{Python}, with
      visualizations using \textbf{matplotlib}, \textbf{pyplot} and \textbf{ggplot}}
    \end{cvitems}
  }

  \projectentry
  {Compositional Semantics in Natural Language Processing and Human Cognition} % Organization
  {\href{https://github.com/jaffourt/Project-Portfolio/tree/main/Project4}{\faicon{github}}} % Location
  {
    \begin{cvitems} % Description(s) of tasks/responsibilities
      \item {Led a research project investigating compositional semantics of \textbf{computational language models}
      and human semantic representations, leveraging high dimensional \textbf{neural data} (fMRI);
      this project investigated unexplained divergence between human cognition and computational language models,
        providing an empirical framework for benchmarking}
      \item{Tested \textbf{natural language processing} model predictions by running interactive online experiments
      on Amazon \textbf{MTurk}, using HTML and JavaScript}
      \item{Implemented \textbf{signal preprocessing} using gaussian mixture models, artifact detection, and
      multivariate statistics on neural \textbf{time series} data in \textbf{MATLAB}}
      \item {Created software to perform statistical tests computing complex similarity rankings between semantic
      classes in \textbf{Python} (NumPy, scikit-learn)}
    \end{cvitems}
  }

  \projectentry
  {Probabilistic Language Atlas \& Data Release Web App} % Organization
  {\href{https://github.com/jaffourt/Project-Portfolio/tree/main/Project1}{\faicon{github}}} % Location
  {
    \begin{cvitems} % Description(s) of tasks/responsibilities
      \item {Led a research project investigating the probabilistic hemodynamic activations across human brains during
      a language task; this project created a foundational tool for researcher's investigations into the neurocognitive
      correlates of language}
      \item {Created software to standardize and preprocess over 300,000 fMRI brain images in two separate
      pipelines, and applying \textbf{linear models} and \textbf{time-series} analyses in \textbf{Matlab},
        \textbf{Python}, and \textbf{R}}
      \item {Designed and deployed RESTful web app on \textbf{Amazon Web Services} for interactive data visualization,
        and open source data distribution in Javascript (\textbf{React}), \textbf{SQL}, and Python (\textbf{Django})
        \href{http://54.84.230.25:3000/}{\faGlobe}}
      \item {Created documentation of data analyses, data processing pipelines, and web app for team use, and maintained
      distributed version control systems in \textbf{Git}}
    \end{cvitems}
  }

%-------------------------------------------------------------------------------
%	JOB
%-------------------------------------------------------------------------------

  \blankcventry
  {Machine Learning Researcher} % Degree
  {Riverside Research} % Institution
  {May 2018 - May 2019} % Location
  {} % Date(s)

  \projectentry
  {Pathway Estimation Using Remotely Sensed Spectral-Terrain Data} % Organization
  {\href{https://github.com/jaffourt/Project-Portfolio/tree/main/Project3}{\faicon{github}}} % Location
  {
    \begin{cvitems} % Description(s) of tasks/responsibilities
      \item {Led a research project which processed high dimensional satellite imagery, and predicted optimal routes
      to navigate the terrain; this project's novel algorithm was presented at Military Operations Research Symposium, 2019}
      \item {Applied novel \textbf{signal processing} and \textbf{machine learning} methods to high dimensional imagery,
        such as \textbf{Principal Component Analysis}, \textbf{Orthogonal Matching Pursuit}, \textbf{Automatic Target
        Generation Procedure}, and \textbf{Non-negative Matrix Factorization} in MATLAB}
      \item {Implemented search algorithms and \textbf{reinforcement learning} to solve an optimization problem for agent-based
      path estimation within a weighted graph in MATLAB}
      \item {Designed and created a \textbf{Graphical User Interface} for interactive 3D visualizations at each step of
      the algorithm in MATLAB}
    \end{cvitems}
  }

  \projectentry
  {Temperature Dependent Tissue Characterization Using VNIR Imagery} % Organization
  {\href{https://github.com/jaffourt/Project-Portfolio/blob/main/Project5/Project5.pdf}{\faicon{github}}} % Location
  {
    \begin{cvitems} % Description(s) of tasks/responsibilities
      \item {Assisted in a research project which characterized temperature dependent tissue samples using Hyper
      Spectral Imagery and machine learning; this project investigated a machine learning approach to automate
      the Maillard reaction in a mass production environment}
      \item {Built and trained a \textbf{deep learning} model using \textbf{competitive leaky learning} to classify
      clusters of distinct spectral vectors}
      \item {Assessed model accuracy by computing the \textbf{Kullback–Leibler divergence}, and Bhattacharyya
      distance between spectral clusters}
    \end{cvitems}
  }

\end{cventries}
